\documentclass{styrdokument}

\title{Stadga}
\date{\today}
%\link{\href{https://github.com/Talmanspresidiet/styrdokument}{\texttt{https://github.com/Talmanspresidiet/styrdokument}}}

\begin{document}

\section{Inledande bestämmelser}

\subsection{Ändamål}

\? Fysikteknologsektionen vid Chalmers studentkår, nedan benämnt sektionen, är en ideell förening för studenter vid utbildningsprogrammen Teknisk fysik, Teknisk matematik och därtill associerade masterprogram vid Chalmers tekniska högskola.

\? Sektionen har till uppgift att verka för sammanhållning mellan medlemmarna och tillvarata deras gemensamma intressen i utbildningsfrågor och studiesociala frågor.

\? Sektionen har även till uppgift att skapa och upprätthålla goda kontakter med Chalmers, sektionen närstående personer och andra sektioner och institutioner samt värna om sektionens traditioner.

\? Sektionen är fackligt, partipolitiskt och religiöst oberoende.

\subsection{Verksamhet}

\? Sektionens räkenskapsår löper från 1 juli till 30 juni.
Ett organ på sektionen får ha annat verksamhetsår enligt reglemente.

\? Sektionsstyrelsen ska ha sitt säte i Göteborg.

\subsection{Skyddshelgon}

\? Sektionens skyddshelgon är Dragos.

\? Sektionens medlemmar vördar Dragos.

\section{Medlemskap}

\subsection{Allmänt}

\? Sektionen har ordinarie medlemmar, hedersmedlemmar, seniormedlemmar, alumner och särskilda medlemmar.

\? Medlem som ska erlägga sektionsavgift, eller summa motsvarande denna, gör detta terminsvis.

\? Rättigheter tillkommer enligt detta kapitel,
\cref{%
  ratt.sektmote.rost,%
  ratt.sektmote.grund,%
  ratt.snf,%
  ratt.intressef,%
  ratt.offentlighet%
}.

\? Medlem är skyldig att rätta sig efter sektionens stadgar, reglemente, beslut samt övriga styrdokument.


\subsection{Ordinarie medlemmar}

\? Ordinarie medlem i sektionen är kårmedlem som studerar vid sektionens program och har betalt sektionsavgift.

\subsubsection{Rättigheter}
\? Ordinarie medlem är valbar till post inom sektionen.

\? Ordinarie medlem har rätt att nyttja av sektionen erbjudna tjänster.

\subsection{Hedersmedlemmar}

\? Sektionens hedersmedlemmar förtecknas i reglementet.

\? Till hedersmedlem kan kallas nu levande person som främjat sektionen, teknologerna vid dess program, ämnesområdet fysik eller matematik; eller på annat sätt tillförskansat sig sektionsmedlemmarnas vördnad och respekt.

\subsubsection{Förslag och kallande}
\? Förslag till hedersmedlem lämnas skriftligen till sektionsstyrelsen och talmanspresidiet med minst 25 namnunderskrifter från medlemmar.

\? Ärendet ska tas upp på nästa sektionsmöte, dock tidigast 6 läsdagar efter att förslaget inkommit.
Beslut om kallande skall fattas med minst \sfrac{2}{3} majoritet.
\label{beslut.heders.kallande}
  
\? Vid bifall kallas personen till nästa sektionsmöte, adjungerad med närvaro- och yttranderätt, där val förrättas.

\? Personen ska närvara vid valet, eller ha inkommit med skriftligt bifall.

\? Beslut om inval ska fattas med minst \sfrac{2}{3} majoritet.
\label{beslut.heders.inval}

\subsubsection{Rättigheter}
\? Hedersmedlem har närvaro- och intaganderätt på alla sektionens arrangemang öppna för samtliga medlemmar.

\subsection{Seniormedlemmar}

\? Seniormedlem är person utnämnd efter ansökan som har skänkt en summa motsvarande sektionsavgiften till sektionen.

\subsubsection{Ansökan}
\? Teknolog vid sektionens program har rätt att efter avslutade eller definitivt avbrutna studier ansöka om seniormedlemskap.
Ansökan lämnas skriftligen till sektionsstyrelsen.

\? Sektionsstyrelsen utnämner seniormedlemmar efter ansökan.

\subsubsection{Rättigheter}
\? Seniormedlem är valbar till post inom sektionen.

\? Seniormedlem har rätt att utnyttja av sektionen erbjudna tjänster.

\? Seniormedlem har närvaro- och intaganderätt på alla sektionens arrangemang öppna för samtliga medlemmar.

\subsection{Alumner}

\? Alumn är den som avlagt master- eller civilingenjörsexamen som ordinarie medlem av sektionen.

\subsection{Särskild medlem}

\? Särskild medlem är kårmedlem vid Chalmers tekniska högskola, utnämnd av sektionsmötet, som har skänkt en summa motsvarande sektionsavgiften till sektionen.

\subsubsection{Ansökan}
\? Kårmedlem vid Chalmers tekniska högskola har rätt att söka till särskild medlem.
Ansökan lämnas skriftligen till sektionsstyrelsen och talmanspresidiet.

\? Ärendet ska tas upp på nästa sektionsmöte, dock tidigast 6 läsdagar efter att förslaget inkommit.
Beslut om utnämnande ska fattas med minst \sfrac{2}{3} majoritet.
\label{beslut.sarskildmedlem}

\subsubsection{Rättigheter}
\? Särskild medlem har samma rättigheter som ordinarie medlem.

\section{Inspektor}
\subsection{Definition}

\? Sektionens inspektor ska vara professor vid institutionen för fysik eller institutionen för matematiska vetenskaper och tillvarata sektionens teknologers intressen, samt fungera som en länk mellan teknologer och anställda.
	
\? Sektionens inspektor väljs på två på varandra följande sektionsmöten med minst \sfrac{2}{3} majoritet, för mandat på tre år.
\label{beslut.inspektor}
	
\? Nominering till inspektor inlämnas till sektionsstyrelsen med minst 25 namnunderskrifter från medlemmar, eller lyfts av sektionsstyrelsen med enkel majoritet.
Innan val skall sektionsstyrelsen inhämta den nominerades samtycke till kandidatur.

\? Inspektors mandat kan förlängas med tre år åt gången av sektionsmötet.
Om mandatet ej förlängs ska nyval av inspektor ske på nästa sektionsmöte.
Nuvarande inspektor kvarstår interimistiskt tills dess att ny inspektor blivit vald.

\? Inspektor har närvaro-, yttrande- och förslagsrätt vid sammanträde i sektionens samtliga organ.

\? Inspektor har rätt att ta del av mötesprotokoll och sektionens övriga handlingar, undantaget de dokument som listas som icke offentliga i reglementet.

\section{Organisation och ansvar}

\subsection{Verksamhetsutövning}

\? Sektionens verksamhet utövas på det sätt denna stadga med tillhörande reglemente föreskriver genom följande organ:
\begin{itemize}
    \item Sektionsmötet
    \item Sektionsstyrelsen
    \item Studienämnden
    \item Kommittéer
    \item Sektionsföreningar
    \item Funktionärer
    \item Intresseföreningar
    \item Talmanspresidiet
    \item Valberedningen
    \item Revisorer
\end{itemize}

\? Alla organ har rätt att i namn och emblem använda sektionens namn och dess symboler.

\? Alla organ är skyldiga att rätta sig efter sektionens stadgar, reglemente, beslut samt övriga styrdokument.

\? Alla organ, med undantag för funktionärer, ska ha representation vid sektionsmötet.

\? Sektionsaktiv är person vald till post av sektionsmötet eller sektionsstyrelsen.

\? Förtroendeposter förtecknas i stadga eller reglemente.
Förtroendevald är person vald till förtroendepost.

\? Ekonomiskt ansvariga förtroendevalda förtecknas i stadga eller reglemente.

\subsection{Ansvarsförhållanden} \label{ansvar}

\? Sektionsmötet är sektionens högsta beslutande organ.

\? Sektionsmötet har till sitt förfogande alla sektionens organ, undantaget intresseföreningar.

\? Sektionsstyrelsen har till sitt förfogande studienämnden, kommittéer, sektionsföreningar och funktionärer, inom deras respektive verksamhetsområden.

\? Sektionsstyrelsen, talmanspresidiet, valberedningen, revisorer och förtroendevalda svarar inför sektionsmötet.
Övriga sektionsaktiva svarar inför sektionsstyrelsen.

\subsubsection{Delegation}
\? Överordnat organ enligt detta avsnitt har rätt att delegera ansvar och uppgifter samt beordra åtgärd.

\subsection{Tillsättande}

\? Post i sektionens organ tillsätts i vanliga fall genom personval på sektionsmöte.
\label{tills.personval}

\? Vid vakans kan sektionsstyrelsen preliminärt tillsätta posten.
Beslutet fastställs på nästkommande sektionsmöte.

\? Kandidat ska närvara vid inval eller lämna skriftligt samtycke till sin kandidatur.

\subsubsection{Valbarhet}
\? Ingen kan samtidigt inneha fler än en förtroendepost, eller fler än en post i samma organ.

\? Ingen kan samtidigt inneha post i fler än en av sektionsstyrelsen, talmanspresidiet, revisorer och valberedning.

\? Talman, vice talman, och ledamot i valberedningen får ej inneha post i studienämnden eller kommitté.

\? Revisor får ej inneha ekonomiskt ansvar inom sektionen.

\? Talmanspresidiet och revisorer behöver ej vara medlemmar av sektionen.

\? Revisor ska vara myndig.

\? Ekonomiskt ansvarig ska vara myndig.

\subsection{Entledigande}

\? Sektionsaktiv kan avsäga sig sin post och på begäran entledigas av sektionsstyrelsen.
Beslutet fastställs på nästkommande sektionsmöte.

\? Sektionsaktiv som upphör som medlem ska entledigas, såvida inte sektionsstyrelsen beslutar annorlunda.
Beslutet fastställs på nästkommande sektionsmöte.

\? Vid entledigande av ekonomiskt ansvarig ska bokslut för perioden fram till fyllnadsvalet upprättas inom fyra veckor från fyllnadsvalet.
\label{entle.bokslut}

\subsubsection{Sektionsstyrelsen}
\? Om ledamot i sektionsstyrelsen entledigas ska extra sektionsmöte utlysas inom 15 läsdagar där fyllnadsval ska äga rum.

\? Om sektionsstyrelsen i sin helhet entledigas ska omedelbart interimstyrelse och ny valberedning väljas.

\? Interimstyrelsen övertar styrelsens befogenheter och skyldigheter tills ny styrelse är vald, men får endast handha löpande ärenden.

\? Om sektionsordförande entledigas tar vice sektionsordförande över som tillförordnad sådan tills ny sektionsordförande är vald.

\subsubsection{Talmanspresidiet}
\? Om talmanspresidiet i sin helhet entledigas sker utlysning av och kallelse till extra sektionsmöte där fyllnadsval äger rum av sektionsordförande.
Sektionsmötet väljer ett temporärt presidium för mötet.

\? Om talmannen entledigas tar vice talman över som tillförordnad sådan tills ny talman är vald.

\subsubsection{Misstroendeförklaring}
\? En högre instans enligt \cref{ansvar} kan förklara att en sektionsaktiv inte har dess förtroende.

\? Yrkande om misstroendeförklaring kan väckas av antingen styrelseledamot, revisor eller 25 medlemmar med förslagsrätt på sektionsmötet.

\? Sådant yrkande ska prövas inom 15 läsdagar.
Om sektionsmötet ska pröva förtroendet ska kallelsen tydligt ange att en förtroendefråga behandlas.
\label{beslut.misstroende.kallelse}

\? Svarande vars förtroende behandlas ska under alla omständigheter beredas möjlighet att närvara och tala för sin sak.

\? Entledigande till följd av misstroende sker med sluten omröstning med minst \sfrac{2}{3} majoritet.
\label{beslut.misstroende.majoritet}

\? Om sektionsstyrelsen beslutar om entledigande till följd av misstroende ska beslutet fastställas på
nästkommande sektionsmöte.

\? Om yrkande om missförtroendeförklaring avser talmanspresidiet i del eller helhet sker utlysning av och kallelse till extra sektionsmöte av sektionsordförande.
Sektionsmötet väljer ett temporärt presidium för mötet.

\? Om ekonomiskt ansvarig entledigas genom misstroendeförklaring beslutar sektionsmötet om bokslutet enligt \cref{entle.bokslut} upprättas av avgående eller nyvald kassör.
De nyvalda ekonomiskt ansvariga anses fria från ansvar för föregående redovisning.

\subsection{Överklagan}

\? Beslut av sektionsmötet eller sektionsstyrelsen som strider mot kårens eller sektionens stadga, reglemente, beslut eller övriga styrdokument får undanröjas efter överklagan.

\? Överklagan gällande kårens styrdokument framställs av kårmedlem till kårfullmäktige.

\? Överklagan gällande sektionens styrdokument framställs av sektionsmedlem till kårfullmäktige, och om det gäller sektionsstyrelsebeslut även sektionsmötet.

\section{Sektionsmötet}
\subsection{Allmänt}

\? Sektionsmötet är sektionens högsta beslutande organ, i vilket samtliga medlemmar har rätt att delta.

\? Fråga av större principiell betydelse ska avgöras av sektionsmötet.

\subsection{Åligganden}

\? Ordinarie sektionsmöte ska sammanträda minst en gång per läsperiod.

\? Det åligger sektionsmötet att innan utgången av varje läsperiod
\begin{itemize}
	\item behandla aktuella årsredovisningar och ansvarsfriheter enligt \cref{arsredovisning},
    \item välja sektionsaktiva enligt reglemente.
\end{itemize}

\? Det åligger sektionsmötet att innan utgången av läsperiod 1
\begin{itemize}
    \item fastställa verksamhetsplan för sektionsstyrelsen,
    \item fastställa budget för innevarande räkenskapsår,
    \item fastställa hur föregående års överskott ska disponeras,
    \item fastställa programrådsledamöter utsedda av sektionsstyrelsen,
    \item vart tredje år välja inspektor.
\end{itemize}

\? Det åligger sektionsmötet att innan utgången av läsperiod 4
\begin{itemize}
    \item välja sektionsordförande, vice sektionsordförande och sektionskassör,
    \item välja talmanspresidium,
    \item välja valberedning,
    \item välja revisorer.
\end{itemize}

\subsection{Utlysning} \label{utlysning}

\? Sektionsmötet sammanträder på kallelse av talman, eller vid vakans på kallelse av sektionsstyrelsen.

\? Ordinarie sektionsmöte utlyses senast 10 läsdagar i förväg genom anslag av preliminär föredragningslista och kallelse enligt reglemente.

\? Slutlig föredragningslista, tillsammans med inkomna handlingar, anslås 3 läsdagar före sammanträdet enligt reglemente.

\? Extra sektionsmöte får begäras av styrelseledamot, inspektor, Chalmers studentkårs styrelse, revisor eller minst 25 medlemmar med förslagsrätt.
Sådant möte ska hållas inom 15 läsdagar.
  
\? Extra sektionsmöte utlyses senast 5 läsdagar i förväg på samma sätt som ordinarie sektionsmöte.

\? Särskilda bestämmelser om kallelse finns i
\cref{%
  beslut.misstroende.kallelse,%
  beslut.andring.kallelse,%
  beslut.upplosning%
}.

\subsection{Beslutsförhet}

\? Sektionsmötet är behörigt om det är utlyst enligt \cref{utlysning} samt fler än 15 röstberättigade medlemmar, ej räknat sektionsstyrelsen och talmanspresidiet, är närvarande.

\? Om färre än 25 röstberättigade medlemmar, ej räknat sektionsstyrelsen och talmanspresidiet, är närvarande, kan beslut endast fattas om ingen yrkar på bordläggning.

\subsection{Sammanträden}

\subsubsection{Rättigheter}
\? Närvaro-, yttrande-, förslags- och rösträtt tillkommer ordinarie medlem samt särskild medlem.
\label{ratt.sektmote.rost}

\? Närvaro-, yttrande-, och förslagsrätt tillkommer inspektor, revisor samt talmanspresidiet.

\? Närvaro- och yttranderätt tillkommer övriga medlemmar, Chalmers studentkårs ledning, Chalmers studentkårs inspektor samt adjungerade.
\label{ratt.sektmote.grund}

\subsubsection{Beslut och votering}
\? Beslut fattas i regel med enkel majoritet.
Vid lika röstetal i sakfråga avgör talmannen.

\? Undantag från beslutsordningen finns i
\cref{%
  beslut.heders.kallande,%
  beslut.heders.inval,%
  beslut.sarskildmedlem,%
  beslut.inspektor,%
  beslut.misstroende.majoritet,%
  beslut.extraarende,%
  beslut.stadgeandring,%
  beslut.reglementesandring,%
  beslut.upplosning%
}.
  
\? Votering sker i regel öppet. Undantag finns i
\cref{beslut.misstroende.majoritet},
samt i reglemente och i mötesordning.

\? Röst får ej avläggas av ombud.

\subsubsection{Motion}
\? Motion kan inlämnas till talmanspresidiet av innehavare av förslagsrätt senast 6 läsdagar i förväg, eller senast 11 läsdagar i förväg om ärendet ska anges i kallelse.

\? Sektionsstyrelsen ges möjlighet att yttra sig i ärendet.

\subsubsection{Extra ärende}
\? Ärende som inte angivits på slutgiltig föredragningslista får tas upp om sektionsmötet med \sfrac{2}{3} majoritet så beslutar och ärendet ej behöver anges i kallelse.
\label{beslut.extraarende}
  
\subsection{Protokoll}
\? Sektionsmötesprotokoll förs enligt \cref{protokoll} och justeras av två av mötet valda justerare.

\? Justerat protokoll ska anslås senast tre läsveckor efter mötet.

\? Sektionsmötesbeslut ej förtecknat i annat styrdokument sammanställs i särskild förteckning.

\section{Sektionsstyrelsen}
\subsection{Definition}

\? Sektionsstyrelsen handhar den verkställande ledningen av sektionens verksamhet och ansvar för sektionens ekonomi.

\? Sektionsstyrelsen består av sektionsordförande, vice sektionsordförande, sektionskassör samt övriga ledamöter enligt reglemente.
Samtliga är förtroendeposter.

\? Sektionsordförande utövar i nödfall sektionsstyrelsens befogenheter.
Ordförandebeslut ska fastställas snarast av sektionsstyrelsen.

\? Vice sektionsordförande övertar i sektionsordförandens frånvaro dennes befogenheter och plikter.

\? Sektionsordförande och sektionskassör är ekonomiskt ansvariga.

\subsection{Styrelsemöte}

\? Sektionsstyrelsen sammanträder på kallelse av sektionsordförande.

\? Sektionsstyrelsen är beslutsför om fler än hälften av sektionsstyrelsens ledamöter är närvarande inklusive sektionsordförande eller vice sektionsordförande.

\? Protokoll förs vid styrelsemöte, justeras av två styrelseledamöter och anslås senast två läsveckor efter mötet.

\section{Studienämnden}
\subsection{Definition}

\? Studienämnden (SNF) har till uppgift att inom sektionen övervaka tillståndet och utvecklingen beträffande studiefrågor, aktivt verka för god kurslitteratur samt främja kontakten med lärarna.

\? Studienämnden består av studienämndsordförande, studienämndskassör och övriga ledamöter enligt reglemente.

\? Studienämndsordförande och studienämndskassör är förtroendeposter och ekonomiskt ansvariga.

\subsection{Studienämndsmöte}

\? Studienämnden sammanträder på kallelse av studienämndsordförande.

\? Ordinarie och särskilda medlemmar har närvaro-, yttrande- och förslagsrätt på studienämndsmöte.
\label{ratt.snf}

\? Endast studienämndens ledamöter har rösträtt på studienämndsmöte.

\? Protokoll förs vid studienämndsmöte, justeras av en studienämndsledamot och anslås senast två läsveckor efter mötet.

\section{Kommittéer}
\subsection{Definition}

\? Kommittéer verkar för att uppfylla sektionens ändamål och ska vara dess representanter.

\? Kommittéer fastställs och förtecknas i reglemente, med syfte, sammansättning och åligganden.

\? Sammansättningen ska åtminstone innehålla ordförande och kassör, vilka är förtroendeposter och ekonomiskt ansvariga.

\section{Sektionsföreningar}
\subsection{Definition}

\? Sektionsföreningar verkar för att uppfylla sektionens ändamål.

\? Sektionsföreningar fastställs och förtecknas i reglemente, med sammansättning och åligganden.

\section{Funktionärer}
\subsection{Definition}

\? Funktionärer verkar för att uppfylla sektionens ändamål genom enskild uppgift.

\? Funktionärer fastställs och förtecknas i reglemente, med sammansättning och åligganden.

\? Om reglemente så föreskriver kan funktionär tillsättas på annat sätt än enligt \cref{tills.personval}.

\section{Intresseföreningar}
\subsection{Definition}

\? Intresseförening är en sammanslutning av sektionsmedlemmar med gemensamt intresse som verkar enligt sektionens ändamål.

\? Intresseföreningstatus beviljas och fråntas av sektionsmötet.

\? Intresseföreningar förtecknas i reglemente.

\subsection{Krav}

\? Intresseförening ska ha en av sektionsstyrelsen godkänd stadga.

\? Intresseförening ska ha en styrelse bestående till minst \sfrac{2}{3} av sektionsmedlemmar.

\? Varje sektionsmedlem ska ha rätt till medlemskap i intresseförening.
Dock får medlem som motverkar intresseföreningens syfte uteslutas.
\label{ratt.intressef}

\? Sektionsmedlemmar ska utgöra minst hälften av intresseförenings medlemmar.

\subsection{Ekonomi och revision}

\? Intresseförening ska ha ekonomi fristående från sektionen.

\? Intresseföreningars verksamhet och ekonomi granskas, utöver egen revision, av sektionens revisorer.

\? Efter att intresseförenings årsmöte behandlat förgående verksamhetsår ska alla handlingar inom tre veckor tillsändas sektionsstyrelsen och sektionens revisorer.

\? Sektionsstyrelsen prövar frågan om ansvarsfrihet för intresseföreningens styrelse efter fullgjort verksamhetsår.

\section{Talmanspresidiet}
\subsection{Definition}

\? Talmanspresidiet leder sektionsmötet och förvaltar sektionens demokratiska
grund.

\? Talmanspresidiet består av talman, vice talman och sekreterare.

\? Vice talman övertar i talmans frånvaro dennes befogenheter och plikter.

\section{Valberedningen}
\subsection{Definition}

\? Valberedningen ansvarar för oberoende nomineringar till samtliga förtroendeposter, inklusive sektionsstyrelsen, och övriga poster enligt reglemente.
  
\? Valberedningen består av 7 ledamöter varav två internt väljs till ordförande respektive vice ordförande.

\? Vid vakans kan sektionsstyrelsen tillförordna valberedningsledamöter tills nästa sektionsmöte.

\subsection{Nomineringsbeslut}

\? Valberedningen sammanträder för nominering på kallelse av ordförande.
Upp till två representanter för organ till vilket ska nomineras får adjungeras.

\? Valberedningen är beslutsför om minst tre ledamöter är närvarande inklusive ordförande eller vice ordförande, samt minst en adjungerad representant för organ till vilket ska nomineras.

\? Nominering ska beslutas senast 7 dagar innan sektionsmötet då val äger rum.

\? Valberedningens nomineringar anslås i enlighet med reglemente samt tillsänds sektionsstyrelsen och talmanspresidiet.

\section{Ekonomi och revision}
\subsection{Allmänt}

\? Sektionens organisationsnummer är 857208-8477.

\? Sektionens firma tecknas av sektionsordförande och sektionskassör var för
sig.

\subsection{Redovisning och ansvarsfrihet} \label{arsredovisning}

\? Sektionsstyrelsen upprättar gemensamt bokslut för sektionen, innefattande alla organ utom intresseföreningar.

\? Studienämnden, kommittéer och övriga organ enligt reglemente upprättar dessutom egen löpande redovising och delbokslut.

\? Sektionsstyrelsen handhar annars den löpande redovisningen.

\? Gemensam årsredovisning enligt detta avsnitt presenteras vid första ordinarie sektionsmöte efter verksamhetsårets slut.
Handlingarna tillställs revisorer, sektionsstyrelsen och talmanspresidiet senast 11 läsdagar innan mötet.

\? Frågan om ansvarsfrihet för förtroendeposter behandlas vid första ordinarie sektionsmöte efter berört organs verksamhetsår av sektionsmötet på grundval av revisionsberättelsen.

\subsection{Revisor}

\? Lekmannarevisorerna granskar oberoende och kontinuerligt sektionens verksamhet och ekonomi.

\? Revisor får ej granska verksamhetsår för organ, då denna var ledamot det året eller föregående, eller då annat jäv föreligger.
\label{revisor.jav}

\? Två ordinare revisorer väljs.
Om revisorer enligt \cref{revisor.jav} inte kan granska samtliga organ väljs extra revisor för dessa.

\? Revisor åligger att avge och anslå revisionberättelse, inklusive yttrande angående ansvarsfrihet, senast 4 läsdagar innan sektionsmöte.

\section{Styrdokument}
\subsection{Allmänt}

\? Utöver denna stadga ska finnas reglemente och övriga styrdokument förtecknade däri.

\? Originalstadgar tillhandahas av sektionsstyrelsen.

\subsection{Ändring}

\? Ändring av stadga eller reglemente görs av sektionsmötet.
Kallelsen ska tydligt ange att fråga om ändring behandlas tillsammans med fullständigt förslag.
\label{beslut.andring.kallelse}

\? Beslut om stadgeändring fattas med \sfrac{2}{3} majoritet.
Fastställande av beslutet görs med \sfrac{2}{3} majoritet på nästkommande sektionsmöte, dock tidigast efter fyra läsveckor förflutit.
Lydelsen får ej ändras vid fastställandet.
\label{beslut.stadgeandring}

\? Stadgeändring ska fastställas av Chalmers studentkårs styrelse.

\? Beslut om reglementesändring fattas med \sfrac{2}{3} majoritet.
\label{beslut.reglementesandring}

\subsection{Protokoll och tillkännagivande} \label{protokoll}

\? Protokoll som förs i sektionens organ ska innehålla anteckningar om ärendenas art, samtliga ställda och ej återtagna yrkanden, beslut samt särskilda yttranden och reservationer.

\? Protokoll och beslut tillkännages genom behörigt anslag enligt reglemente.

\? Beslut fattade inom sektionen som berör Chalmers studentkår i dess helhet meddelas Chalmers studentkårs styrelse.

\? Medlem har rätt att ta del av mötesprotokoll och sektionens övriga handlingar, undantaget de dokument som listas som icke offentliga i reglemente.
\label{ratt.offentlighet}

\subsection{Tolkningstvister}

\? Vid konflikt med annat styrdokument har stadga och reglemente företräde.
Vid konflikt med reglemente har stadga företräde.

\? Vid tvist om stadgans tolkning avgörs frågan av inspektor.

\section{Upplösning}
\subsection{Upplösningsbeslut}

\? Sektionen upplöses genom beslut på två sektionsmöten, mellan vilka minst 4 läsveckor förflutit, med \sfrac{3}{4} majoritet och minst 25 bifallande medlemmar.
Kallelsen ska tydligt ange att fråga om upplösning behandlas.
\label{beslut.upplosning}

\subsection{Avveckling}

\? Vid upplösning avvecklar sektionsstyrelsen snarast sektionen.
Eventuellt överskott tillfaller Chalmers studentkår.
Sektionens handlingar arkiveras på Göteborgs föreningsarkiv.

\? Chalmers studentkår förvaltar eventuellt överskott i fond fram tills ny sektion bildats för studerande på utbildningsprogrammet för Teknisk fysik och/eller Teknisk matematik eller motsvarande.

\end{document}
